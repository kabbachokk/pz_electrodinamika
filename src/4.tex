% !TEX program = xelatex
\documentclass[fontsize=14pt,a4paper]{scrartcl}
\usepackage[english,russian]{babel} 
\usepackage{fontspec} 
\defaultfontfeatures{Ligatures={TeX},Renderer=Basic} 
\setmainfont[Ligatures=TeX]{Times New Roman}

\usepackage{unicode-math}
\unimathsetup{math-style=TeX}
\setmathfont{Cambria Math}
\setmathfont[range=\mathup/{num}]{Times New Roman}
\setmathfont[range=\mathit/{greek,Greek,latin,Latin,Cyrillic,cyrillic}]{Cambria Math}
\setmathfont[range=\mathup/{greek,Greek,latin,Latin,Cyrillic,cyrillic}]{Cambria Math}
\setmathfont[range={"2212,"002B,"003D,"0028,"0029,"005B,
"005D,"221A,"2211,"2248,"222B,"007C,"2026,"2202,"00D7,"0302,
"2261,"0025,"22C5,"00B1,"2194,"21D4}]{Cambria Math}

\usepackage{indentfirst}
\usepackage{graphicx}
\usepackage{amsmath}

\usepackage{pgfplots}
\usepackage{pgfplotstable}

\def\arraystretch{1.3}%

\usepackage[left=20mm, top=20mm, right=10mm, bottom=20mm, nohead, footskip=10mm]{geometry}
\begin{document}
  \begin{titlepage}                                                         
    \newpage                                                                        
    \begin{center}   
      Министерство образования и науки Российской Федерации  \\ 
      \vspace{1em}                                                    
      {\mdseries
        Федеральное государственное бюджетное образовательное \\
        учреждение высшего образования \\
        «ОМСКИЙ ГОСУДАРСТВЕННЫЙ ТЕХНИЧЕСКИЙ УНИВЕРСИТЕТ»
      }                               
      \vspace{1em}      

      {\bfseries Кафедра «Комплексная защита информации»}  

      \vspace{\fill}                                                         
                                   
      {\bfseries ОТЧЕТ } \\                                 
      По дисциплине «Электродинамика и распространение радиоволн» \\ 
      \vspace{1em} 
      Практическая работа №4 \\                                                           
    \end{center}                                                          
                                                                                        
    \vspace{\fill}                                                         
                                                                                        
    \hfill\parbox{5cm}{
      Выполнили\\
      студенты гр. БИТ-181:\\
      Белый В.Е., \\
      Шабанов В.С.\\
      \\
      Проверил:\\
      доц., канд. физ-мат.н. \\
      Михеев В.В.\\
    }                                                                                                                              
                                                                                                                                                                              
    \vspace{\fill}                                                    
                                                                                        
    \begin{center}                                                        
    Омск, 2020                                                                
    \end{center}                                                          
                                                                                        
    \end{titlepage}

    \newpage
    {\bfseries Задание 4.} 
    Оценить возможность использования прямоугольного $(axb, a=2,1b)$ и круглого $(\varnothing = a)$ волноводов на заданной частоте $f$.
    \\ \indent Рассчитать одномодовый и рабочий диапазон частот, найти все типы мод на заданной частоте $f$, предложить (при необходимости) волновод оптимальных размеров.
    \\ \indent Рассчитать основные характеристики (затухание, $\lambda_\textup{В}$, $\upsilon_\textup{гр}$, $\upsilon_\textup{ф}$, $\textup{Р}_\textup{пред}$) для волноводов оптимальных размеров из заданного материала ($\sigma_{Cu} = 5,7\cdot10^7 \; \textup{См/м}$) при коэффициенте шероховатости $k_\textup{ш}$.    

    \begin{table}[ht!]
      \begin{center}
        \label{tab:table1}
        \begin{tabular}{|l|l|l|l|l|l|l|}
          \hline
          Вариант & $a, \textup{см}$  & $f, \textup{ГГц}$ & $k_\textup{ш}$  & $\sigma, \textup{См/м}$   & $\mu, \textup{См/м}$ & $\mu_{11}$    \\
          \hline
          $45$    & $7,0$             & $6,0$             & $1,4$           & $5,7\cdot10^7$            & $1$                  & $1,8412$      \\
          \hline
        \end{tabular}
        \caption{Исходные данные.}
      \end{center}
    \end{table}

    {\bfseries Решение:} 
    \\ \indent Для прямоугольного волновода рассчитаем:
    \begin{itemize}
      \item{
        одномодовый режим
        \begin{equation} f_{\textup{кр}}^{H_{10}} = \frac{c}{2a} < f_{\textup{одн. мод.}} < f_{\textup{кр}}^{H_{20}} = \frac{c}{a} \end{equation}
        \begin{equation} 2,14\cdot 10^9 \; \textup{Гц} < f_{\textup{одн. мод.}} < 4,29\cdot 10^9 \; \textup{Гц}; \end{equation}    
      }
      \item{
        рабочий диапазон частот \\
        \begin{equation} 1,25f_{\textup{кр}}^{H_{10}} = \frac{1,25c}{2a} < f_{\textup{раб.}} < 0,99f_{\textup{кр}}^{H_{20}} = \frac{1,98c}{2a} \end{equation}
        \begin{equation} 2,68\cdot 10^9 \; \textup{Гц} < f_{\textup{раб.}} < 4,24\cdot 10^9 \; \textup{Гц}; \end{equation}
      }
    \end{itemize}

    \indent Для круглого волновода рассчитаем:
    \begin{itemize}
      \item{
        одномодовый режим
        \begin{equation} f_{\textup{кр}}^{H_{11}} = \frac{8,8\cdot 10^7}{a} < f_{\textup{одн. мод.}} < f_{\textup{кр}}^{H_{20}} = \frac{11,5\cdot 10^7}{a} \end{equation}
        \begin{equation} 1,43\cdot 10^9 \; \textup{Гц} < f_{\textup{одн. мод.}} < 1,63\cdot 10^9 \; \textup{Гц}; \end{equation}
   }
      \item{
        рабочий диапазон частот \\
        \begin{equation} \frac{10^8}{a} < f_{\textup{раб.}} < \frac{11,4\cdot 10^7}{a} \end{equation}
        \begin{equation} 1,43\cdot 10^9 \; \textup{Гц} < f_{\textup{раб.}} < 1,63\cdot 10^9 \; \textup{Гц}; \end{equation}
      }
    \end{itemize}
    \newpage
    \indent Рассчитаем оптимальные размеры волноводов на заданной частоте $f$:
    \begin{itemize}
      \item{
        для прямоугольного волновода \\
        \begin{equation} \frac{0,63c}{f} < a < \frac{0,99c}{f} \end{equation}
        \begin{equation} 0,0315 \; \textup{м} < a < 0,0495 \; \textup{м}; \end{equation} 
        $a=0,0405 \; \textup{м}$ - среднее значение от вычисленных пределов;   
      }
      \item{
        для круглого волновода \\
        \begin{equation} \frac{1\cdot10^8}{f} < a < \frac{1,14\cdot10^8}{f} \end{equation}
        \begin{equation} 0,017 \; \textup{м} < a < 0,019 \; \textup{м}; \end{equation}      
        $a=0,018 \; \textup{м}$ - среднее значение от вычисленных пределов; 
      }
    \end{itemize}

    \indent Рассчитаем  основные характеристики для для прямоугольного волновода оптимальных размеров ($a=0,0405$):
    \begin{itemize}
      \item{
        критическая частота
        \begin{equation}
          f_{\textup{кр}} = \frac{c}{2a}, \quad f_{\textup{кр}} = 3,7\cdot10^9 \; \textup{Гц};
        \end{equation}
      }
      \item{
        апертура волновода
        \begin{equation}
          \sqrt{K} = \sqrt{1-\left(\frac{f_{\textup{кр}}}{f}\right)^2}, \quad\sqrt{K} = 0,787;
        \end{equation}
      }
      \item{
        фазовая и групповая скорости
        \begin{gather}
          \upsilon_\textup{ф} = \frac{c}{\sqrt{K}}, \quad\upsilon_\textup{ф} = 3,812\cdot10^8 \; \textup{м/c};\\
          \upsilon_\textup{гр} = c\sqrt{K}, \quad\upsilon_\textup{гр} = 2,361\cdot10^8 \; \textup{м/c};
        \end{gather}
      }
      \item{
        коэффициенты затухания и фазы
        \begin{gather}
          \alpha=\sqrt{\pi f \sigma \mu_0 \mu}, \quad\alpha=1,162\cdot10^6 \; \textup{1/м};\\
          \beta=\frac{2\pi f}{c}\sqrt{K}, \quad\beta=98,897 \; \textup{1/м};
        \end{gather}
      }
      \item{
        глубина проникновения
        \begin{equation}
          \Delta^\circ = \frac{1}{\alpha}, \quad\Delta^\circ =8,605\cdot 10^{-7} \; \textup{м};
        \end{equation}
      }
      \item{
        длина волны генератора и длина волны в волноводе
        \begin{gather}
          \lambda=\frac{c}{f}, \quad\lambda=0,05 \; \textup{м};\\
          \lambda_\textup{В}=\frac{\lambda}{\sqrt{K}}, \quad\lambda_\textup{В}=0,064 \; \textup{м};
        \end{gather}
      }
      \item{
        волновое число
        \begin{equation}
          k=\frac{2\pi}{\lambda}, \quad k=125,664 \; \textup{1/м};
        \end{equation}
      }
      \item{
        коэффициент затухания волны H-типа
        \begin{equation}
          \alpha_\textup{пр}=\frac{k_\textup{ш}k\mu_\textup{пр}\Delta^\circ}{b}\cdot\left[\frac{1}{\sqrt{K}}\cdot\left(\frac{1}{2} - \frac{b}{a}\cdot\frac{f_{\textup{кр}}^2}{f^2}\right)\right], \quad\alpha_\textup{пр}=1,023\cdot10^{-3} \; \textup{1/м};
        \end{equation}
      }
      \item{
        предельная мощность ($E_\textup{проб}^2=30 \; \text{кВ/см}$)
        \begin{equation}
          P_\textup{пред}=\frac{ab}{4}\cdot\frac{\sqrt{K}}{Z_\textup{В}}\cdot E_\textup{проб}^2, \quad P_\textup{пред}=4,004\cdot 10^8 \; \textup{Вт};
        \end{equation}
      }
    \end{itemize}

    \indent Рассчитаем  основные характеристики для для круглого волновода оптимальных размеров ($a=0,018$):
    \begin{itemize}
      \item{
        критическая частота
        \begin{equation}
          f_{\textup{кр}} = \frac{8,8\cdot 10^7}{a}, \quad f_{\textup{кр}} = 4,89\cdot10^9 \; \textup{Гц};
        \end{equation}
      }
      \item{
        апертура волновода
        \begin{equation}
          \sqrt{K} = \sqrt{1-\left(\frac{f_{\textup{кр}}}{f}\right)^2}, \quad\sqrt{K} = 0,579;
        \end{equation}
      }
      \item{
        фазовая и групповая скорости
        \begin{gather}
          \upsilon_\textup{ф} = \frac{c}{\sqrt{K}}, \quad\upsilon_\textup{ф} = 5,181\cdot10^8 \; \textup{м/c};\\
          \upsilon_\textup{гр} = c\sqrt{K}, \quad\upsilon_\textup{гр} = 1,737\cdot10^8 \; \textup{м/c};
        \end{gather}
      }
      \item{
        коэффициенты затухания и фазы
        \begin{gather}
          \alpha=\sqrt{\pi f \sigma \mu_0 \mu}, \quad\alpha=1,162\cdot10^6 \; \textup{1/м};\\
          \beta=\frac{2\pi f}{c}\sqrt{K}, \quad\beta=72,759 \; \textup{1/м};
        \end{gather}
      }
      \item{
        глубина проникновения
        \begin{equation}
          \Delta^\circ = \frac{1}{\alpha}, \quad\Delta^\circ =8,605\cdot 10^{-7} \; \textup{м};
        \end{equation}
      }
      \item{
        длина волны генератора и длина волны в волноводе
        \begin{gather}
          \lambda=\frac{c}{f}, \quad\lambda=0,05 \; \textup{м};\\
          \lambda_\textup{В}=\frac{\lambda}{\sqrt{K}}, \quad\lambda_\textup{В}=0,086 \; \textup{м};
        \end{gather}
      }
      \item{
        волновое число
        \begin{equation}
          k=\frac{2\pi}{\lambda}, \quad k=125,664 \; \textup{1/м};
        \end{equation}
      }
      \item{
        коэффициент затухания волны H-типа
        \begin{equation}
          \alpha_\textup{пр}=\frac{k_\textup{ш}k\mu_\textup{пр}\Delta^\circ}{2a\sqrt{K}}\cdot\left[\left(\frac{f_{\textup{кр}}}{f}\right)^2 + \frac{n^2}{\mu_{nm}^2 - n^2} \right], \quad \alpha_\textup{пр}=7,263\cdot10^{-3} \; \textup{1/м};
        \end{equation}
      }
      \item{
        предельная мощность ($E_\textup{проб}^2=30 \; \text{кВ/см}$)
        \begin{equation}
          P_\textup{пред}=1,990\cdot 10^{-3}a^2\sqrt{K}\frac{Z_0}{Z_\text{В}}E_{проб.}^2, \quad P_\textup{пред}=3,36\cdot10^8 \; \textup{Вт};
        \end{equation}
      }
    \end{itemize}

    {\bfseries Вывод:} 
    \\ \indent При выполнении задания были рассчитаны одномодовые диапазоны частот, оптимальные размеры волноводов и параметры для прямоугольного и круглого волновода.
    \begin{table}[ht!]
      \begin{center}
        \label{tab:table2}
        \begin{tabular}{|l|l|l|}
          \hline
          Параметр & Прямоугольный & Круглый \\
          \hline
          $\upsilon_\textup{ф},\textup{м/c}$    & $3,812\cdot10^8$    & $5,181\cdot10^8$    \\
          \hline
          $\upsilon_\textup{гр}, \textup{м/c}$  & $2,361\cdot10^8$    & $1,737\cdot10^8$    \\
          \hline
          $\sqrt{K}$                            & $0,787$             & $0,579$             \\
          \hline
          $\alpha_\textup{пр}, \text{1/м}$      & $1,023\cdot10^{-3}$ & $7,263\cdot10^{-3}$ \\
          \hline
          $P_\textup{пред}, \text{Вт}$          & $4,004\cdot10^8$    & $3,36\cdot10^8$     \\
          \hline
        \end{tabular}
        \caption{Сравнение параметров волноводов.}
      \end{center}
    \end{table}
\end{document}