% !TEX program = xelatex
\documentclass[14pt,a4paper]{scrartcl} 
\usepackage[utf8]{inputenc}
\usepackage[english,russian]{babel} 
\usepackage{fontspec} 
\defaultfontfeatures{Ligatures={TeX},Renderer=Basic} 
\setmainfont[Ligatures=TeX]{Times New Roman}

\usepackage{unicode-math}
\unimathsetup{math-style=TeX}
\setmathfont{Cambria Math}
\setmathfont[range=\mathup/{num}]{Times New Roman}
\setmathfont[range=\mathit/{greek,Greek,latin,Latin,Cyrillic,cyrillic}]{Cambria Math}
\setmathfont[range=\mathup/{greek,Greek,latin,Latin,Cyrillic,cyrillic}]{Cambria Math}
\setmathfont[range={"2212,"002B,"003D,"0028,"0029,"005B,
"005D,"221A,"2211,"2248,"222B,"007C,"2026,"2202,"00D7,"0302,
"2261,"0025,"22C5,"00B1,"2194,"21D4}]{Cambria Math}

\usepackage{indentfirst}
\usepackage{graphicx}
\usepackage{amsmath}

\def\arraystretch{1.3}%

\usepackage[left=20mm, top=20mm, right=10mm, bottom=20mm, nohead, footskip=10mm]{geometry}
\begin{document}
  \begin{titlepage}                                                         
    \newpage                                                                        
    \begin{center}   
      Министерство образования и науки Российской Федерации  \\ 
      \vspace{1em}                                                    
      {\mdseries
        Федеральное государственное бюджетное образовательное \\
        учреждение высшего образования \\
        «ОМСКИЙ ГОСУДАРСТВЕННЫЙ ТЕХНИЧЕСКИЙ УНИВЕРСИТЕТ»
      }                               
      \vspace{1em}      

      {\bfseries Кафедра «Комплексная защита информации»}  

      \vspace{\fill}                                                         
                                   
      {\bfseries ОТЧЕТ } \\                                 
      По дисциплине «Электродинамика и распространение радиоволн» \\ 
      \vspace{1em} 
      Практическая работа №2 \\                                                           
    \end{center}                                                          
                                                                                        
    \vspace{\fill}                                                         
                                                                                        
    \hfill\parbox{5cm}{
      Выполнили\\
      студенты гр. БИТ-181:\\
      Белый В.Е., \\
      Шабанов В.С.\\
      \\
      Проверил:\\
      доц., канд. физ-мат.н. \\
      Михеев В.В.\\
    }                                                                                                                              
                                                                                                                                                                              
    \vspace{\fill}                                                    
                                                                                        
    \begin{center}                                                        
    Омск, 2020                                                                
    \end{center}                                                          
                                                                                        
    \end{titlepage}

    \newpage
    {\bfseries Задание 2.} Плоская гармоническая ЭМВ с частотой $𝑓$, располагается вдоль оси $z$ в проводящей среде с параметрами $\varepsilon=1$, $\mu$, $\sigma$. Амплитуда вектора $E$ в начале координат равна $E_m$.\\
    \indent Найти $\tan{\delta}$, коэффициент затухания и фазы $\lambda_\textup{в}$, $\upsilon_\textup{ф}$, $\upsilon_\textup{гр}$, волновое сопротивление среды, глубину проникновения ЭМВ в вещество. Определить амплитуду плотности тока проводимости и смещения, а также плотность потока мощности волны в начале координат и на расстоянии $z$ от начала координат. Рассчитать, на каком расстоянии от начала координат амплитуда поля уменьшится в $m$ раз.\\
    \indent Сравнить полученные результаты с заданием 1, сделать выводы.

    \begin{table}[h!]
      \begin{center}
        \label{tab:table1}
        \begin{tabular}{|l|l|l|l|l|l|l|}
          \hline
          Вариант & $f$, МГц & $\mu$ & $\sigma$, См/м & $E_m$, В/м & $z$, мм & $m$ \\
          \hline
          $45$ & $1,2$ & $8,0$ & $1\cdot{10}^{6}$ & $1\cdot{10}^{-1}$ & $2,4$ & $5000$ \\
          \hline
        \end{tabular}
        \caption{Исходные данные.}
      \end{center}
    \end{table}

    \begin{table}[h!]
      \begin{center}
        \label{tab:table2}
        \begin{tabular}{|l|l|l|}
          \hline
          $\varepsilon_0$, Ф/м & $\mu_0$, Гн/м & $Z_0$, Ом \\
          \hline
          $\frac{{10}^{-9}}{36\pi}$ & $4\pi\cdot10^{-7}$ & $120\pi$ \\
          \hline
        \end{tabular}
        \caption{Постоянные величины.}
      \end{center}
    \end{table}
    {\bfseries Решение:} \\
    \indent Найдем тангенс потерь, для того чтобы определить классификацию среды.

    \begin{equation} tg\delta=\frac{\sigma}{2\cdot\pi\cdot\varepsilon\cdot f \cdot\varepsilon_{0}}=1,5\cdot10^{10}; \end{equation}

    Поскольку, значение $tg\delta$ находится в интервале от 10 до \infty, можно классифицировать среду как – проводник.
    \begin{equation}\omega = 2\pi\cdot f = 2\pi\cdot 1,2 \cdot 10^{6} = 7,54 \cdot 10^{6} \; \textup{рад/c};\end{equation}
    \begin{equation}\alpha=\beta=\sqrt{\pi f \sigma \mu_0 \mu} = 6,156 \cdot 10^{3} \; \textup{1/м};\end{equation}

    \newpage
    Находим характеристики ЭМВ:
    \begin{itemize}
      \item{
        длину волны: 
        \begin{equation}\lambda=\frac{2\pi}{\beta} = 1,02\cdot 10^{-3} \; \textup{м};\end{equation}
      }
      \item{
        фазовую скорость, групповую скорость: \\
        \begin{equation}\upsilon_\textup{ф} = \frac{\omega}{\beta} = 1,225\cdot 10^{3} \; \textup{м/с}, \; \upsilon_\textup{гр} = 0\; \textup{м/с};\end{equation}
      }
      \item{
        волновое сопротивление проводника:
        \begin{equation}Z_\textup{в}=\sqrt{\frac{\omega \mu_0 \mu}{\sigma}}\exp{\left(\textup{i}\frac{\pi}{4}\right)} = 0.00615+0.00615i \; \textup{Ом};\end{equation}
      }
      \item{
        глубину проникновения:
        \begin{equation}\Delta^\circ = \frac{1}{\alpha} = 1,624\cdot 10^{-4} \; \textup{м};\end{equation}
      }
    \end{itemize}

    Рассчитаем амплитуду плотности тока проводимости и смещения: \\
    \begin{equation}j_\textup{пр} = \sigma E_m = 1\cdot 10^{5} \; \textup{А/м}^{2},\end{equation}
    \begin{equation}j_\textup{см} = \frac{j_\textup{пр}}{tg\delta} = 6,667 \cdot 10^{-6} \; \textup{А/м}^{2};\end{equation}

    Рассчитаем плотность потока мощности ЭМВ: \\
    \begin{equation}\textup{П}_0 = \frac{E^2_m}{2 \cdot Z_\textup{в}} = 0,524 \; \textup{Вт/м}^{2},\end{equation}
    \begin{equation}\textup{П}(z) = \textup{П}_0 \cdot e^{-2\cdot\alpha z} = 8,434\cdot 10^{-14} \; \textup{Вт/м}^{2};\end{equation}

    Рассчитаем на каком расстоянии от начала координат амплитуда поля уменьшается в 5000 раз: \\
    \begin{equation}A = e^{\alpha z} \rightarrow m = e^{\alpha z} \rightarrow z = \frac{\ln{m}}{\alpha} = 1,384\cdot 10^{-3}\; \textup{м};\end{equation}
    \newpage

    {\bfseries Вывод: }\\
    \indent Сравнительный анализ ЭМВ в полупроводящей среде и в проводнике:

    \begin{table}[h!]
      \begin{center}
        \label{tab:table3}
        \begin{tabular}{|l|l|l|}
          \hline
          Параметры                             & Полупроводящая среда  & Проводник             \\
          \hline
          $tg\delta$                            & 0,27                  & $1,5\cdot10^{10}$     \\
          \hline
          $\alpha$, 1/м                         & 0,193                 & $6,156 \cdot 10^{3}$  \\
          \hline
          $\beta$, 1/м                          & 1,634                 & $6,156 \cdot 10^{3}$  \\
          \hline
          $\lambda$, м                          & 3,845                 & $1,02\cdot 10^{-3}$   \\
          \hline
          $\upsilon_\textup{ф}$, м/с            & $3,845\cdot 10^{7}$   & $1,225\cdot 10^{3}$   \\
          \hline
          $\upsilon_\textup{гр}$, м/с           & $3,845\cdot 10^{7}$   & 0                     \\
          \hline
          $Z_\textup{в}$, Ом                    & 47,70720-5,29937i     & 0,00615+0,00615i      \\
          \hline
          $\Delta^\circ$, м                     & 5,181                 & $1,624\cdot 10^{-4}$  \\
          \hline
          $j_\textup{пр}$, $\textup{А/м}^{2}$   & $18 \cdot 10^{-5}$    & $1\cdot 10^{5}$       \\
          \hline
          $j_\textup{см}$, $\textup{А/м}^{2}$   & $6,667 \cdot 10^{-4}$ & $6,667 \cdot 10^{-6}$ \\
          \hline
          $\textup{П}(z)$, $\textup{Вт/м}^{2}$  & $1,24\cdot 10^{-10}$  & $8,434\cdot 10^{-14}$ \\
          \hline
          $z$, м                                & 22,013                & $1,384\cdot 10^{-3}$  \\
          \hline
        \end{tabular}
        \caption{Сравнение параметров.}
      \end{center}
    \end{table}

    Из таблицы 3 можно увидеть, что ...
\end{document}