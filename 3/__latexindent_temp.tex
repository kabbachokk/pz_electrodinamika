% !TEX program = xelatex
\documentclass[14pt,a4paper]{scrartcl} 
\usepackage[utf8]{inputenc}
\usepackage[english,russian]{babel} 
\usepackage{fontspec} 
\defaultfontfeatures{Ligatures={TeX},Renderer=Basic} 
\setmainfont[Ligatures=TeX]{Times New Roman}

\usepackage{unicode-math}
\unimathsetup{math-style=TeX}
\setmathfont{Cambria Math}
\setmathfont[range=\mathup/{num}]{Times New Roman}
\setmathfont[range=\mathit/{greek,Greek,latin,Latin,Cyrillic,cyrillic}]{Cambria Math}
\setmathfont[range=\mathup/{greek,Greek,latin,Latin,Cyrillic,cyrillic}]{Cambria Math}
\setmathfont[range={"2212,"002B,"003D,"0028,"0029,"005B,
"005D,"221A,"2211,"2248,"222B,"007C,"2026,"2202,"00D7,"0302,
"2261,"0025,"22C5,"00B1,"2194,"21D4}]{Cambria Math}

\usepackage{indentfirst}
\usepackage{graphicx}
\usepackage{amsmath}

\usepackage{pgfplots}
\usepackage{pgfplotstable}

\def\arraystretch{1.3}%

\usepackage[left=20mm, top=20mm, right=10mm, bottom=20mm, nohead, footskip=10mm]{geometry}
\begin{document}
  \begin{titlepage}                                                         
    \newpage                                                                        
    \begin{center}   
      Министерство образования и науки Российской Федерации  \\ 
      \vspace{1em}                                                    
      {\mdseries
        Федеральное государственное бюджетное образовательное \\
        учреждение высшего образования \\
        «ОМСКИЙ ГОСУДАРСТВЕННЫЙ ТЕХНИЧЕСКИЙ УНИВЕРСИТЕТ»
      }                               
      \vspace{1em}      

      {\bfseries Кафедра «Комплексная защита информации»}  

      \vspace{\fill}                                                         
                                   
      {\bfseries ОТЧЕТ } \\                                 
      По дисциплине «Электродинамика и распространение радиоволн» \\ 
      \vspace{1em} 
      Практическая работа №3 \\                                                           
    \end{center}                                                          
                                                                                        
    \vspace{\fill}                                                         
                                                                                        
    \hfill\parbox{5cm}{
      Выполнили\\
      студенты гр. БИТ-181:\\
      Белый В.Е., \\
      Шабанов В.С.\\
      \\
      Проверил:\\
      доц., канд. физ-мат.н. \\
      Михеев В.В.\\
    }                                                                                                                              
                                                                                                                                                                              
    \vspace{\fill}                                                    
                                                                                        
    \begin{center}                                                        
    Омск, 2020                                                                
    \end{center}                                                          
                                                                                        
    \end{titlepage}

    \newpage
    {\bfseries Задание 3.} 
    Плоская ЭМВ (поляризация неизвестна) наклонно падает из диэлектрика с параметрами $ \varepsilon $ (табл. 1), $\mu=1$, $\sigma=1$ на плоскую границу раздела с вакуумом.
    \\ \indent Рассчитать значение критического угла и угла Брюстера. Описать условия полного отражения и полного прохождения.
    \\ \indent Построить графики зависимостей модулей коэффициентов отражения $\left(\lvert \textup{Г}(φ)\lvert\right)$ и преломления $\left(\lvert \textup{T}(φ)\lvert\right)$ от угла падения $\left(\varphi=0…90^{\circ}\right)$.    

    \begin{table}[h!]
      \begin{center}
        \label{tab:table1}
        \begin{tabular}{|l|l|}
          \hline
          Вариант & $\varepsilon$   \\
          \hline
          $45$    & $4,4$           \\
          \hline
        \end{tabular}
        \caption{Исходные данные.}
      \end{center}
    \end{table}

    {\bfseries Решение:} 
    \\ \indent В случае, когда ЭМВ падает из оптически более плотной среды в менее плотную, возникает явление полного отражения, если угол падения превышает критический угол. Если угол падения больше критического, то отраженная волна уносит всю энергию, принесенную падающей ЭМВ. При угле Брюстера коэффициент отражения превращается в ноль, ЭМВ полностью переходит во вторую среду.
    \\ \indent Рассчитываем значение критического угла и угла Брюстера, при котором отраженная волна отсутствует:
    
    \begin{center}
      \begin{tabular}{cc}
        $\varphi_{\textup{кр}} = arcsin{\sqrt{\frac{\varepsilon_2\cdot\mu_2}{\varepsilon_1\cdot\mu_1}}},$   & $\varphi_{\textup{Бр}} = arctg\sqrt{\frac{\varepsilon_2}{\varepsilon_1}},$       \\
        $\varphi_{\textup{кр}} = 28,47^{\circ};$                                                                         & $\varphi_{\textup{Бр}} = 25,49^{\circ};$                                                      \\
      \end{tabular}
    \end{center}

    Рассчитаем угол преломления, описанный законом Снеллиуса: \\

    \[ \frac{\sin\varphi}{\sin\psi}=\frac{\sqrt{\varepsilon_2}}{\sqrt{\varepsilon_1}}, \]
    \[ \psi = arcsin\left(\frac{\sin\varphi\cdot\sqrt{\varepsilon_1}}{\sqrt{\varepsilon_2}}\right); \]

    \begin{table}[h!]
      \begin{center}
        \label{tab:table2}
        \begin{tabular}{|l|l|}
          \hline
          $\varphi^\circ$ & $\psi^\circ$ \\
          \hline
          $0$     & $0$     \\
          \hline
          $5$     & $2,38$  \\
          \hline
          $10$    & $4,75$  \\
          \hline
          $15$    & $7,09$  \\
          \hline
          $20$    & $9,38$  \\
          \hline
          $25$    & $11,62$ \\
          \hline
          $30$    & $13,79$ \\
          \hline
          $35$    & $15,87$ \\
          \hline
          $40$    & $17,84$   \\
          \hline
          $45$    & $19,7$ \\
          \hline
          $50$    & $22,99$  \\
          \hline
          $55$    & $21,42$ \\
          \hline
          $60$    & $24,38$ \\
          \hline
          $65$    & $25,6$  \\
          \hline
          $70$    & $26,61$ \\
          \hline
          $75$    & $27,42$ \\
          \hline
          $80$    & $28$    \\
          \hline
          $85$    & $28,35$ \\
          \hline
          $90$    & $28,47$ \\
          \hline
        \end{tabular}
        \caption{Угол по закону Снеллиуса.}
      \end{center}
    \end{table}

    \newpage
    Рассчитаем значения коэффициентов отражения и преломления для перпендикулярной и параллельной поляризации:\\

    \[ \textup{Г}_{\perp} = \frac{\sin(\psi-\varphi)}{\sin(\psi+\varphi)}, \; \textup{Г}_{\parallel} = \frac{\tg(\psi-\varphi)}{\tg(\psi+\varphi)}; \]
    \[ \textup{Т}_{\perp} = \frac{2\cdot\sin\psi\cdot\cos\varphi}{\sin(\psi+\varphi)}, \; \textup{Т}_{\parallel} = \frac{2\cdot\sin\psi\cdot\cos\varphi}{\sin{\left(\psi+\varphi\right)}\cdot\cos(\psi-\varphi)}; \]

    \begin{center}
      \pgfplotstabletypeset[
        col sep=semicolon,
        /pgf/number format/read comma as period
      ]{data.csv}
    \end{center}
    
    \newpage
    Построим графики зависимостей модулей коэффициентов отражения $\left(\lvert \textup{Г}(\varphi)\lvert\right)$ и преломления $\left(\lvert \textup{Г}(φ)\lvert\right)$ и преломления $\left(\lvert \textup{T}(\varphi)\lvert\right)$: \\
    
    \begin{center}
    \begin{tikzpicture}[
      declare function={ 
          psi(\x) = asin((sin(\x)*4.4^(1/2))/(1^(1/2)));
        },
      ]
      \begin{axis}[
        legend pos = north west, 
        title = График зависимостей модулей коэффициентов отражения,
        xlabel = {$\varphi$},
        domain = 0:28.47,
        restrict x to domain = 0:28.47,
        restrict y to domain = 0:1,
        xmin = 0,
        xmax = 40,
        ymin = 0,
        ymax = 1,
        samples = 900
      ]
        \legend{ 
          $\lvert \textup{Г}_{\perp} \lvert$, 
          $\lvert \textup{Г}_{\parallel} \lvert$
        };
        \addplot[blue] {
          abs(sin(psi(x)-x)/sin(psi(x)+x))
        };
        \addplot[red] {
          abs(tan(psi(x)-x)/tan(psi(x)+x))
        };
      \end{axis}
    \end{tikzpicture}
    \begin{tikzpicture}[
      declare function={ 
          psi(\x) = asin((sin(\x)*4.4^(1/2))/(1^(1/2))); 
        },
      ]
      \begin{axis}[
        legend pos = north west, 
        title = График зависимостей модулей коэффициентов отражения,
        xlabel = {$\varphi$},
        domain = 0:28.47,
        restrict x to domain = 0:28.47,
        restrict y to domain = 0:1,
        xmin = 0,
        xmax = 40,
        ymin = 0,
        samples = 900
      ]
        \legend{ 
          $\lvert \textup{Г}_{\perp} \lvert$, 
          $\lvert \textup{Г}_{\parallel} \lvert$
        };
        \addplot[blue] {
          (2*sin(psi(x))*sin(x))/sin(psi(x)+x)
        };
        \addplot[red] {
          (2*sin(psi(x))*sin(x))/(sin(psi(x)+x)+sin(psi(x)-x))
        };
      \end{axis}
    \end{tikzpicture}
    \end{center}
\end{document}